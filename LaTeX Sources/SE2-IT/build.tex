\section{Build}

\subsubsection{Requirements}

\begin{itemize}
    \item Java SE JDK 17 (OracleJDK, OpenJDK)
    \item Maven framework version 3.0 (or newer)
\end{itemize}

\subsection{Windows}
\subsubsection{Installing Java}
Download OracleJDK or OpenJDK.\\
Extract content from the zip folder to your preferred location, then go to

\begin{minted}{bash}
Start>Edit the system environment variables>Environment Variables
\end{minted}
In the User Section select Path variable and click on Edit; select New and type \mintinline{text}{C:\Users\<your-user>\<path-to-extracted-folder>\bin}, then save and exit.
Open \mintinline{text}{Start>cmd.exe} and verify your Java version by typing the following command:
\begin{minted}{text}
    java -version
\end{minted}

\subsubsection{Installing Maven}
Download Maven.\\
Extract to preferred location and repeat above steps.\\
Verify your Maven version by typing the following command:
\begin{minted}{text}
    mvn --version
\end{minted}

\subsection{Linux}
If you want to download Maven and JavaJDK with your package manager be sure to fulfill the system requirements otherwise follow the next steps.
\subsubsection{Installing Java}
Download JavaJDK based on your distro or OpenJDK 17 from official websites.\\
Navigate to your preferred install location and extract the .tar.gz archive using:
\begin{minted}{text}
    tar zxvf jdk-17.<version-number>-x64_bin.tar.gz
\end{minted}
Now let's set the PATH variable by typing the following code:\\
\begin{minted}{bash}

    cd $HOME
    nano .bashrc

\end{minted}

Add the following line to the end of .bashrc:

\begin{minted}{bash}
    export PATH=/<path-to-extracted-folder/bin:$PATH

\end{minted}

Verify your Java version by typing the following command:
\begin{minted}{text}
    java -version
\end{minted}

\subsubsection*{Debian-Based Linux Platforms}
Type:

\begin{minted}{text}
    sudo apt install /path/to/package/name.deb
\end{minted}


Verify your Java version by typing the following command:

\begin{minted}{text}
    java -version
\end{minted}

\subsubsection*{RPM-Based Linux Platforms}
Type the following command to install the package:
\begin{minted}{text}
    rpm -ivh jdk-17.<version-number>-x64_bin.rpm 
\end{minted}
Upgrade the package using the following command:
\begin{minted}{text}
    rpm -Uvh jdk-17.<version-number>-x64_bin.rpm
\end{minted}
You can now delete the .rpm file and verify your installation by typing:
\begin{minted}{text}
    java -version
\end{minted}
\subsubsection{Installing Maven}
Download Maven, extract it with:
\begin{minted}{text}
    tar zxvf maven.<version-number>-x64_bin.tar.gz
\end{minted}
Set PATH variable by appending this command to .bashrc as previous steps:
\begin{minted}{bash}
    cd $HOME
    nano .bashrc
    export PATH=/<path-to-extracted-folder/bin:$PATH
\end{minted}

\subsection{MacOS}

\subsubsection{Installing Java}
Download JavaJDK 17, double-click on .dmg file and Install it. Verify your Java version by typing the following command:
\begin{minted}{text}
    java -version
\end{minted}

\subsubsection{Installing Maven}
Follow Linux installation steps.

\subsection{Compiling the source files}
Open the terminal and navigate to the root application folder (containing pom.xml).
Type 
\begin{minted}{text}
    mvn package -Dmaven.test.skip=true
\end{minted}
 and hit Enter.
 
 \subsection{Run tests}
 In order to run tests move to the home directory of the project (ForumApp ot DataApp) and run
 \begin{minted}{bash}
     mvn test
 \end{minted}